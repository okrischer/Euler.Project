\documentclass{scrreprt}
\KOMAoptions{parskip=half*}
\usepackage{mathtools}
\numberwithin{equation}{chapter}
\usepackage{hyperref}
\hypersetup{hidelinks}
\usepackage[chapter,newfloat=true]{minted}
\usemintedstyle[haskell]{lovelace}
\newminted[code]{haskell}{bgcolor=bgh}
\newminted[spec]{haskell}{bgcolor=bgh}
\usemintedstyle[crystal]{autumn}
\newminted[crystal]{crystal}{bgcolor=bgc}

\begin{document}
\definecolor{bgh}{rgb}{0.96,0.96,0.98}
\definecolor{bgc}{rgb}{0.95,0.98,0.95}
\title{Project Euler}
\author{Oliver Krischer}
\subtitle{Solutions to the problems in Haskell and Crystal}
\maketitle
\begin{abstract}
  Functional approaches to the solution of a problem are implemented in \emph{Haskell} (light blue background), while imperative approaches are implemented in \emph{Crystal} (light green background).

  For most of the solutions simple benchmarks are provided.
  Please keep in mind that those benchmarks have no absolute meaning, as no effort was taken to standarize the benchmarks results of both languages.
  So, you can never say that \emph{Haskell} is faster than \emph{Crystal} or vice versa.

  But, within a given language, the results have a relative meaning: if \mintinline{haskell}{someFunction} is 2.5 times faster than \mintinline{haskell}{anotherFunction} in the same language, then \mintinline{haskell}{anotherFunction} is indeed faster.
\end{abstract}
\tableofcontents

\input{Problem001.lhs}
\input{Problem002.lhs}
\input{Problem003.lhs}
\input{Problem004.lhs}

\end{document}
